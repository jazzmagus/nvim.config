
\RequirePackage{etex}
\documentclass[12pt, a4paper]{exam}
% \reserveinserts{256}
\usepackage[T1]{fontenc}
\usepackage[italian]{babel}
\usepackage{mdframed}
% \usepackage{fontspec}
\usepackage{helvet}
% \usepackage{enumitem}
% \usepackage{hyperref}
\RequirePackage{etex}
\usepackage{tcolorbox}
\usepackage{color,soul}
\usepackage{transparent}
\usepackage{nicefrac}
%\usepackage{fancyhdr}
%\usepackage{float}
% \usepackage[margin=1in]{geometry}
\usepackage[top=2cm, bottom=1.4cm, left=1.4cm, right=2.0cm, includefoot]{geometry}
\usepackage{amsmath, amssymb, amsfonts, mathrsfs}
\usepackage{multicol}
\usepackage{graphicx}
\usepackage{xcolor}
\usepackage{fix-cm}
\usepackage{array}
\usepackage{booktabs}
\usepackage{longtable}
\usepackage{multirow}
\usepackage{setspace}
\usepackage{dashrule}
\usepackage{tikz}
\usetikzlibrary{shapes.geometric,fit}
\usepackage{xparse}
\usepackage{tkz-euclide}
\usepackage{upquote}
\usepackage{caption}
\usepackage{parskip}
\usepackage{longtable}
\usepackage{pgfplots}
\usepackage{enumerate}
\usepackage{fontawesome}
% \usepackage{linegoal}
\usepackage[font=footnotesize,labelfont=bf, format=hang]{caption}
% \usepackage{venndiagram}
\usepackage{booktabs}
\usepackage{colortbl}
\usetikzlibrary{arrows.meta,bending,positioning}
\pgfplotsset{every x tick label/.append style={font=\footnotesize, yshift=0.6ex}, grid style={dashed}}
\pgfplotsset{every y tick label/.append style={font=\footnotesize, xshift=0.5ex}}
\usetikzlibrary{arrows}
\setlength{\columnsep}{1cm}
\tikzset{
  arrow/.style = {-stealth[bend]},
}

\pgfplotsset{compat=1.18}


\renewcommand{\questionshook}{%
  \setlength{\leftmargin}{0pt}%
}
\renewcommand{\choiceshook}{%
  \setlength{\leftmargin}{20pt}%
}

\checkboxchar{$\square$}
\checkedchar{$\boxtimes$}
\CorrectChoiceEmphasis{}  % solo per non avere bold nelle risposte corrette

\renewcommand{\checkboxeshook}{
  \setlength{\labelsep}{2.4em}
  \setlength{\leftmargin}{4em}
}

\newcommand{\ChoiceLabel}[1]{\hspace{-1.6em}\makebox[1.6em][l]{\textbf{#1}}\ignorespaces}
\newcommand{\WrongChoice}[1]{\choice \ChoiceLabel{#1}}
\newcommand{\RightChoice}[1]{\correctchoice \ChoiceLabel{#1}}
\newcommand{\Item}[1]{\hspace{-17pt}\makebox[17pt][l]{\textbf{#1.}}\ignorespaces}
\newcommand{\midmatch}{\hspace{0.75in}\underline{\hspace{0.25in}     }}


\setlength{\parskip}{0.6mm}
% \pgfplotsset{compat=1.18}

% ========================== colors =======================

\definecolor{mygray}{gray}{0.9}
\definecolor{maroon}{cmyk}{0,0.87,0.68,0.32}
\definecolor{almond}{rgb}{0.94, 0.87, 0.8}
\definecolor{cosmiclatte}{rgb}{1.0, 0.97, 0.91}
\definecolor{eggshell}{rgb}{0.94, 0.92, 0.84}
\definecolor{floralwhite}{rgb}{1.0, 0.98, 0.94}
\definecolor{darksienna}{rgb}{0.24, 0.08, 0.08}
\definecolor{darkorange}{rgb}{1.0, 0.55, 0.0}
\definecolor{cinnabar}{rgb}{0.89, 0.26, 0.2}
\definecolor{caribbeangreen}{rgb}{0.0, 0.8, 0.6}
\definecolor{aliceblue}{rgb}{0.94, 0.97, 1.0}
\definecolor{burlywood}{rgb}{0.87, 0.72, 0.53}
\definecolor{neonorange}{rgb}{255, 151, 24}


% ========================== watermark =======================

\usepackage[firstpage]{draftwatermark}
\SetWatermarkLightness{ 0.9 }
\SetWatermarkColor{caribbeangreen!30}
% \SetWatermarkText{\includegraphics[width = 90mm]{welcome.png}}
\SetWatermarkText{soluzioni}
\SetWatermarkScale{ 1.3 }

% ---------------------------------- Intestazione

\newcommand{\class}{\LARGE {PROGRAMMAZIONE di MATEMATICA}}
\newcommand{\term}{\Large {\em Classi QUARTE SE}}
\newcommand{\examnum}{\large{anno scolastico 2023/'24}}
\newcommand{\examtitle}{\large{prof. Diego Fantinelli}}
\newcommand{\examdate}{28 ottobre 2022}
\newcommand{\timelimit}{60 minuti}
\CorrectChoiceEmphasis{\color{maroon}}
\SolutionEmphasis{\color{maroon} \footnotesize}
\renewcommand{\solutiontitle}{\noindent\textbf{Soluzione:}\par\noindent}

\pagestyle{headandfoot}
\firstpageheader{}{}{}
\runningheader{\footnotesize IIS "G. A. Remondini" - Bassano del Grappa (VI)}{}{Programmazione di Matematica}
\runningheadrule

\firstpagefooter{}{}{}
\runningfooter{}{}{pag. \thepage\ di \numpages}
\runningfootrule

% ---------------------------------- Definition of circles

% \def\firstcircle{ (0.0, 0.0) circle (1.5)}
% \def\secondcircle{(2.0, 0.0) circle (1.5)}
% \def\thirdcircle{ (1.0,-1.5) circle (1.5)}
% \def\rectangle{ (-1.5,-3.0) rectangle (3.5,1.0) }
% \colorlet{circle edge}{black}
% \colorlet{circle area}{blue!30}

% \tikzset{filled/.style={fill=circle area, draw=circle edge, thick},
%   outline/.style={draw=circle edge, thick}}

% \setlength{\parskip}{1.2mm}

% % =============================================== Punteggi

% \pointpoints{punto}{\em punti}
% \pointformat{[{\footnotesize \thepoints}]}
% \bonuspointpoints{punto bonus}{\em p.ti bonus}
% \bonuspointformat{[{\footnotesize \thepoints}]}
% \pointsinrightmargin
% \setlength{\rightpointsmargin}{.5cm}
% \chqword{Esercizio}
% \chpword{Punti}
% \chbpword{Punti Bonus}
% \chsword{Punteggio}
% \chtword{Totale}

\begin{document}

% =============================================== Title Page ================================ 

\tikz [remember picture, overlay] %
\node [shift={(1.3cm,-0.5cm)}] at (current page.north west) %
[anchor=north west] %
{\includegraphics[width = 22mm]{logo_rem.jpeg}};\\[2ex]

\begin{tikzpicture}[remember picture, overlay]
  \node[align=left, text width=14cm, anchor=north west] (texy)
    at ($(current page.north) + (-7cm,-0.6cm)$)
    {
    \begin{center}
      \footnotesize {\bf ISTITUTO D’ISTRUZIONE SUPERIORE “G. A. REMONDINI”}\\ TECNICO PER IL TURISMO, LE BIOTECNOLOGIE SANITARIE E LA LOGISTICA PROFESSIONALE PER I SERVIZI COMMERCIALI, SERVIZI PER LA SANITÀ E L’ASSISTENZA SOCIALE,
      SERVIZI PER L'ENOGASTRONOMIA\\
      Via Travettore, 33 - 36061 Bassano del Grappa (VI)\\ tel: 0424.523592 - Fax : 0424.220037 -  VIIS01700L@istruzione.it
    \end{center}
    };
\end{tikzpicture}
% \rule[2ex]{\textwidth}{0.5pt}\\
\tikz [remember picture, overlay] %
\node [shift={(17.5cm,-0.5cm)}] at (current page.north west) %
[anchor=north west] %
{\includegraphics[width = 18mm]{logo_italy.png}};\\[2ex]

\begin{center}
\vfill
\rule[2ex]{\textwidth}{0.5pt}
{\bf \class}\\[20pt]
{\bf \term }\\[30pt]

\begin{multicols}{2}
{\em \large{\examtitle}}\\
{\em \large{\examnum}}
\end{multicols}

\rule[2ex]{\textwidth}{0.5pt}\\
\end{center}

% \vspace{1.5cm}
% \begin{tabular*}{\textwidth}{l @{\extracolsep{\fill}} r @{\extracolsep{6pt}} l}
% \textbf{} & \textbf{COGNOME e Nome:} & \makebox[2.5in]{\hrulefill}\\
% \textbf{} &&\\
% \textbf{} & \textbf{Classe:} & \makebox[2.5in]{\Large{\bf 3 \string^ QES}}\\
% \textbf{} &&\\
% \textbf{} & data: & \makebox[2.5in]{\examdate}\\
% \textbf{} &&\\

% \textbf{} & Tempo a disposizione: & \makebox[2.5in]{\timelimit}\\
% \textbf{} &&\\
% \textbf{} &&\\
% \textbf{} &&\\
% % \textbf{} & {\em prof.:} & \makebox[2.5in]{\em Diego Fantinelli}\\

% % \textbf{} &&\\
% % \textbf{} &&\\
% \textbf{} &&\\
% \textbf{} & {\em voto finale:} & \makebox[2.5in]{\fillin}\\[20pt]

% \end{tabular*}\\

% \noindent
% $\star$ {\em eventuali osservazioni e/o considerazioni del docente}:
% \vspace{0pt}
% \fillwithdottedlines{0.5in}

\vfill


% =============================================== Avvertenze =================================

% \small{\textbf{Istruzioni e avvertenze:}}
% \small{
% \SetSinglespace{1ex}{
% \begin{itemize}
% \item La presente verifica, somministrata in modalità {\em in presenza}, contiene \numquestions \, quesiti, per un totale di \numpoints \, punti, più $2$ punti bonus; l'esercizio facoltativo che verrà conteggiato ove siano già stati risolti tutti i precedenti.
% % \item Per la parte riguardante il TEST:
% % \begin{itemize}
% %     \item Le risposte vanno accuratamente riportate nella {\bf Tabella delle Risposte} allegata;
% %     \item Ogni risposta esatta vale $2$ punti;
% %     \item Le risposte che richiedono una giustificazione valgono $2$ punto soltanto se corrette e complete;
% %     \item Ogni risposta errata o non data vale $0$ punti;
% %     \item Per modificare una risposta è sufficiente cerchiare quella errata e segnare nuovamente quella corretta.
% % \end{itemize}
% \item {\bf La sufficienza è fissata a $14$ punti}

% \item Il voto verrà riportato in capo alla presente verifica, e sarà oggetto di un confronto costruttivo con lo studente.
% \item Eventuali copiature palesi comporteranno l'annullamento della prova e un voto pari a 3, a prescindere dal punteggio totalizzato.
% % \item La sufficienza è fissata a $30$ punti, ma potrebbe subire delle modifiche in fase di correzione al fine di garantire la validità della prova anche in caso di prestazioni lontane dalla media-classe auspicata.
% \item È vietato l'utilizzo di calcolatrici scientifiche, smartphone, tablet e altri dispositivi digitali, così come l'accesso a internet, nonché la consultazione di testi, appunti e/o siti web, ove non preventivamente autorizzato.
% \end{itemize}
% }
% }

% \printanswers


% % ============================== ESERCIZI ==============================

\newpage
% \vspace{5pt}

% % ============================== Esercizi


\section*{\LARGE{Programmazione Classi Quarte SE}}
\vspace{2em}

\hrule
\bigskip
\subsection*{TEMA 0 - RIPASSO}

\begin{quote}
\textbf{periodo}: settembre
\end{quote}


\subsubsection*{conoscenze}

\begin{itemize}

\item
  Disequazioni numeriche intere di primo grado.
\end{itemize}


\subsubsection*{abilità}

\begin{itemize}

\item
  Risolvere disequazioni di primo grado e verificare la correttezza dei
  procedimenti utilizzati.
\end{itemize}

\bigskip
\hrule
\bigskip

\subsection*{TEMA 1 - EQUAZIONI DI PRIMO GRADO FRATTE}

\begin{quote}
\textbf{periodo}: ottobre-novembre \textbf{competenza}: A1, A2
\end{quote}


\subsubsection*{conoscenze}

\begin{itemize}
\item
  Frazioni algebriche: semplificazioni e operazioni con le frazioni
  algebriche (moltiplicazione e divisione)
\item
  Equazioni numeriche di primo grado fratte.
\item
  Tecniche risolutive di un problema, anche utilizzando equazioni di
  primo grado.
\end{itemize}


\subsubsection*{abilità}

\begin{itemize}
\item
  Risolvere espressioni con le frazioni algebriche.
\item
  Risolvere equazioni di primo grado e verificare la correttezza dei
  procedimenti utilizzati.
\item
  Utilizzo dell'algebra per risolvere problemi numerici.
\end{itemize}

\bigskip
\hrule
\bigskip

\subsection*{Tema 2 - RADICALI, EQUAZIONI DI 2° GRADO}

\begin{quote}
\textbf{periodo}: novembre-dicembre

\textbf{competenze}: A1, A2
\end{quote}


\subsubsection*{conoscenze}

\begin{itemize}
\item
  Definizione di radicale e le sue condizioni di esistenza
\item
  La proprietà invariantiva
\item
  Operazioni con i radicali
\item
  Regole risolutive delle equazioni di secondo grado, complete e
  incomplete.
\item
  Significato e discussione del discriminante di un'equazione di 2°
  grado.
\item
  Equazioni di secondo grado intere e fratte.
\end{itemize}

\
\subsubsection*{abilità}

\begin{itemize}
\item
  Semplificare espressioni utilizzando le operazioni con i radicali.
\item
  Risolvere equazioni di secondo grado numeriche intere e fratte e
  verificare la correttezza dei procedimenti utilizzati.
\item
  Risolvere semplici problemi di secondo grado
\end{itemize}

\bigskip
\hrule
\bigskip

\subsection*{Tema 3 - GEOMETRIA ANALITICA: RETTA}

\begin{quote}
\textbf{periodo}: gennaio-febbraio \textbf{competenze}: B1, B2
\end{quote}


\subsubsection*{conoscenze}

\begin{itemize}
\item
  Il \textbf{piano cartesiano}: distanza tra due punti, punto medio di
  un segmento.
\item
  La retta nel piano cartesiano: retta passante per l'origine, retta in
  posizione generica, significato geometrico del coefficiente angolare e
  di ordinata all'origine, rette parallele e perpendicolari,
  intersezione tra rette, equazione di retta passante per due punti.
\end{itemize}


\subsubsection*{abilità}

\begin{itemize}
\item
  Saper rappresentare nel piano cartesiano una retta nota la sua
  equazione e determinare l'equazione di una retta note alcune
  condizioni.
\item
  Problemi di scelta tra più alternative (solo a livello di
  esercitazione)
\end{itemize}

\bigskip
\hrule
\bigskip

\subsection*{Tema 4 - SISTEMI DI EQUAZIONI e PROBLEMI}

\begin{quote}
\textbf{periodo}: marzo-aprile \textbf{competenze}: A1, A2
\end{quote}


\subsubsection*{conoscenze}

\begin{itemize}
\item
  Sistemi di equazioni di primo grado.
\item
  Interpretazione geometrica dei sistemi di equazioni.
\item
  Conoscere le regole per risolvere un problema con equazioni o sistemi
  di primo grado.
\end{itemize}


\subsubsection*{abilità}

\begin{itemize}
\item
  Saper risolvere un sistema di primo grado con diversi metodi:
  sostituzione, addizione e sottrazione, metodo grafico.
\item
  Risolvere problemi che implicano l'uso di funzioni, di equazioni e di
  sistemi di equazioni anche per via grafica, collegati con altre
  discipline e situazioni di vita ordinaria, come primo passo verso la
  modellizzazione matematica.
\end{itemize}

\bigskip
\hrule
\bigskip

\subsection*{Tema 5 - GEOMETRIA ANALITICA: LA PARABOLA}

\begin{quote}
\textbf{periodo}: febbraio-marzo \textbf{competenze}: B1, B2
\end{quote}


\subsubsection*{conoscenze}

\begin{itemize}
\item
  La parabola nel piano cartesiano: parabola con asse di simmetria
  parallelo all'asse y, studio dell'equazione \[y=ax^2+bx+c\] con casi
  particolari, formule del vertice e dell'asse di simmetria.
\item
  Disegnare il grafico della parabola dopo aver determinato:

  \begin{itemize}
  \item
    vertice,
  \item
    asse di simmetria,
  \item
    intersezioni con gli assi.
  \end{itemize}
\end{itemize}

\subsubsection*{abilità}

\begin{itemize}
\item
  Stabilire algebricamente e graficamente posizione retta-parabola.
\item
  Saper rappresentare graficamente nel piano cartesiano la parabola nota
  l'equazione.
\item
  Saper determinare le intersezioni tra retta e parabola.
\end{itemize}

\bigskip
\hrule
\bigskip

\subsection*{Tema 6 - DISEQUAZIONI DI II GRADO}

\begin{quote}
\textbf{periodo}: maggio-giugno \textbf{competenze}: B2
\end{quote}

\subsubsection*{conoscenze}

\begin{itemize}
\item
  Disequazioni di 2°grado intere e fratte (risoluzione grafica).
\item
  Sistemi di disequazioni di secondo grado
\end{itemize}


\subsubsection*{abilità}

\begin{itemize}
\item
  Acquisire le tecniche per la risoluzione grafica di disequazioni di 2°
  grado.
\end{itemize}

\bigskip
\hrule
\bigskip

\subsection*{Obiettivo finale - Competenze}

\begin{itemize}
  \item Saper utilizzare le tecniche di calcolo per risolvere le equazioni di 1° grado fratte,  quelle di 2° grado e i sistemi di equazioni lineari.
  \item Saper risolvere problemi di geometria analitica sulla retta.
  \item Saper risolvere problemi di geometria analitica sulla  parabola.
  \item Saper risolvere disequazioni di 1° e 2°  facendo uso della retta e della parabola.
\end{itemize}
% \begin{questions}
% % ------------------------------ Esercizio 1
% \addpoints

% \question
% Determina il valore numerico delle seguenti espressioni in \( \mathbb{N} \):\\

% \begin{parts}

% \part[4]

% \( \left\{\left[\left(15^5: 5^5\right)^2: 3^8\right]^5: 3^7-5^2\right\}^{11}:\left[\left(2^4 \cdot 2^6\right)^3:\left(2^{22} \cdot 2^0\right)\right] \)

% \fillwithdottedlines{0.25in}


% \begin{solution}
% \( [ 8 ] \)	
% \end{solution}

% \vspace{20pt}

% \part[4]

% \( \left[\left(3^2 \cdot 2^2\right)^5: 6^8\right]-\left\{\left[\left(16^3\right)^2:\left(2^5\right)^4\right] \cdot 2^9\right\}: 4^5-\left(2^4 \cdot 5^2\right)^0 \)

% \fillwithdottedlines{0.25in}

% \begin{solution}
% \( [ 27 ] \)	
% \end{solution}

% \end{parts}

% \vspace{10pt}
% \rule[.1ex]{\textwidth}{0.2pt}

% % ------------------------------ Esercizio 2

% \addpoints

% \question[6] determina \(\textbf{M.C.D.} \) e \(\textbf{m.c.m.} \) dei seguenti gruppi di numeri, dopo averli opportunamente scomposti in fattori:
% \begin{multicols}{2}

% \begin{parts}

% \part
%   \( 81, \; 54 , \; 72 \);
% \fillwithdottedlines{0.25in}

% \begin{solution}
%   \( [\text{M.C.D.} = 3^2 \quad \text{m.c.m.} = 2^3 \cdot 3^4]\)
% \end{solution}

% \part
%   \( 63, \; 90, \; 30 \);
% \fillwithdottedlines{0.25in}

% \begin{solution}
%   \( [\text{M.C.D.} = 3 \quad \text{m.c.m.} = 2 \cdot 3^2 \cdot 5 \cdot 7]\)
% \end{solution}

% \end{parts}
% \end{multicols}

% \vspace{10pt}
% \rule[.1ex]{\textwidth}{0.2pt}


% % ------------------------------ Esercizio 3 

% \addpoints

% \question[6] il valore delle seguenti potenze di potenze, facendo particolare attenzione alle regole dei segni:
% \begin{multicols}{2}

% \begin{parts}

% \part
%   \( - \biggl\{- \left[ - \left(-1\right)^{3}\right]^{3} \biggr \}^{3} \)
% \fillwithdottedlines{0.25in}

% \begin{solution}
%   \( [+ 1]\)
% \end{solution}

% \part
%   \( - \biggl\{ \left[ - 1^{2}\right]^{3} \biggr \}^{3} \)

% \fillwithdottedlines{0.25in}

% \begin{solution}
%   \( [+ 1]\)
% \end{solution}

% \end{parts}
% \end{multicols}

% \vspace{20pt}
% \rule[.1ex]{\textwidth}{0.2pt}

% % ------------------------------ Esercizio 4
% % * modificare l'esercizio: numeri Interi

% \addpoints
% \question[6] Determina il valore numerico delle seguente espressione in \( \mathbb{Z} \):\\

% % \begin{parts}
% % \part[4]

% \( \left[(-3) \cdot(-5)+(-2)^5:(-2)^2\right] \cdot 3^2+\left(3^3 \cdot 2^2\right)+\left[(-7)^2:(-7) \cdot 3^2\right] \)

% \fillwithdottedlines{0.25in}

% \begin{solution}
%   \( \begin{aligned}
%     &{\left[(-3) \cdot(-5)+(-2)^5:(-2)^2\right] \cdot 3^2+\left(3^3 \cdot 2^2\right)+\left[(-7)^2:(-7) \cdot 3^2\right]=} \\
%     &=\left[+15+(-2)^{5-2}\right] \cdot 9+(27 \cdot 4)+\left[(-7)^{2-1} \cdot 9\right]= \\
%     &=\left[+15+(-2)^3\right] \cdot 9+108+\left[(-7)^1 \cdot 9\right]= \\
%     &=[+15+(-8)] \cdot 9+108+[-56]= \\
%     &=[15-8] \cdot 9+108+[-56]= \\
%     &=7 \cdot 9+108-56= \\
%     &=56+108-56= \\
%     &=164-56=108
%     \end{aligned} \)
% \end{solution}

% \vspace{10pt}

% \vfill
% % %\newpage
% % \vspace{30pt}
% \rule[.1ex]{\textwidth}{0.2pt}

% % --------------------------------- Esercizio 6 - facoltativo
% \addpoints
% \bonusquestion[2] {\em Esercizio facoltativo:} Determina il valore numerico delle seguente espressione in \( \mathbb{Z} \):\\

% \( -(-2)^6:(-2)^4+(-6)^2:(-3)^2+\left[(+2)^3 \cdot(-2)^3\right]: 2^3 \)

% \fillwithdottedlines{0.25in}

% \begin{solution}
%   \(
%     \begin{aligned}
%       &-(-2)^6:(-2)^4+(-6)^2:(-3)^2+\left[(+2)^3 \cdot(-2)^3\right]: 2^3= \\
%       &=-(-2)^{6-4}+(-6:(-3))^2+[8 \cdot(-8)]: 8= \\
%       &=-(-2)^2+(+2)^2+[-64]: 8= \\
%       &-2+4+(-8)= \\
%       &=2-8=-6
%     \end{aligned} \)	
% \end{solution}

% \end{questions}

% % \vfill
% \newpage
% % \rule[.2ex]{\textwidth}{0.2pt}
% \hrule width \hsize \kern 1mm \hrule width \hsize height 2pt

% % ================================ Tabella Punteggi
% % todo: rivedere punteggio

% \begin{multicols}{2}

% \subsection*{Tabella dei punteggi}
% \vspace{5pt}
% \begin{center}

% \combinedgradetable[v][questions]
% \end{center}
% \vspace{10pt}
% \footnotesize{La sufficienza è fissata a $14$ punti}
% % \vspace{40pt}
% \vfill
% % \columnbreak

% \begin{center}
% \subsection*{Griglia di valutazione}
% \end{center}
% \vspace{5pt}
% {\color{cinnabar}
% \begin{center}

% \large{
  
%   \begin{tabular}{c|c}
%   \toprule
    
%     \bf{punteggio} &  \bf{voto}\\[5pt]
%     \midrule
%     $< 8$ & $4$ \\[5pt]
    
%     $8$ & $4\frac{1}{2}$ \\[5pt]
    
%     $10$ & $5$ \\[5pt]
    
%     $12$ & $5\frac{1}{2}$ \\[5pt]
    
%     \Large{\textbf{$14$}} & \Large{\textbf{$6$}} \\[5pt]
      
%     $16$ & $6 \frac{1}{2}$ \\[5pt]
    
%     $18$ & $7$ \\[5pt]
    
%     $20$ & $7 \frac{1}{2}$ \\[5pt]
    
%     $22$ & $8$ \\[5pt]
    
%     $24$ & $8 \frac{1}{2}$ \\[5pt]
    
%     $26$ & $9$ \\[5pt]
    
%     $28 - \text{\em bonus}$ & $9 \frac{1}{2}$ \\[5pt]

% \bottomrule
%   \end{tabular}
%   }
% \end{center}
% }

% \end{multicols}


% \newpage
% \begin{center}
% \begin{tikzpicture}
% \draw[step=.3cm,burlywood,very thin] (0,0) grid (17.7,26.1);
% \end{tikzpicture}
% \end{center}

\vfill


% ============== firma =======================================

\begin{tabular}{p{0.64\textwidth}p{0.3\textwidth}}
    Bassano del Grappa, \today  & Diego Fantinelli \\[8pt]
    & \hrulefill \\
\end{tabular}

\end{document}

