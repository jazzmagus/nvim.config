\RequirePackage{etex}
\documentclass[12pt, a4paper]{exam}
% \reserveinserts{256}
\usepackage[T1]{fontenc}
\usepackage[italian]{babel}
\usepackage{mdframed}
% \usepackage{fontspec}
\usepackage{helvet}
% \usepackage{enumitem}
\usepackage{hyperref}
\usepackage{tcolorbox}
\usepackage{color,soul}
\usepackage{transparent}
\usepackage{nicefrac}
%\usepackage{fancyhdr}
%\usepackage{float}
% \usepackage[margin=1in]{geometry}
\usepackage[top=1.8cm, bottom=1.4cm, left=1.4cm, right=2.0cm, includefoot]{geometry}
\usepackage{amsmath, amssymb, amsfonts, mathrsfs}
\usepackage{multicol}
\usepackage{graphicx}
\usepackage{xcolor}
\usepackage{fix-cm}
\usepackage{array}
\usepackage{booktabs}
\usepackage{longtable}
\usepackage{multirow}
\usepackage{setspace}
\usepackage{dashrule}
\usepackage{tikz}
\usetikzlibrary{shapes.geometric,fit}
\usepackage{xparse}
\usepackage{tkz-euclide}
\usepackage{upquote}
\usepackage{caption}
\usepackage{parskip}
\usepackage{longtable}
\usepackage{pgfplots}
\usepackage{enumerate}
\usepackage{fontawesome}
% \usepackage{linegoal}
\usepackage[font=footnotesize,labelfont=bf, format=hang]{caption}
% \usepackage{venndiagram}
\usepackage{booktabs}
\usepackage{colortbl}
\usetikzlibrary{arrows.meta,bending,positioning}
\pgfplotsset{every x tick label/.append style={font=\footnotesize, yshift=0.6ex}, grid style={dashed}}
\pgfplotsset{every y tick label/.append style={font=\footnotesize, xshift=0.5ex}}
\usetikzlibrary{arrows}
\setlength{\columnsep}{1cm}
\tikzset{
  arrow/.style = {-stealth[bend]},
}

\pgfplotsset{compat=1.18}


\renewcommand{\questionshook}{%
  \setlength{\leftmargin}{0pt}%
}
\renewcommand{\choiceshook}{%
  \setlength{\leftmargin}{20pt}%
}

\checkboxchar{$\square$}
\checkedchar{$\boxtimes$}
\CorrectChoiceEmphasis{}  % solo per non avere bold nelle risposte corrette

\renewcommand{\checkboxeshook}{
  \setlength{\labelsep}{2.8em}
  \setlength{\leftmargin}{4em}
}

\newcommand{\ChoiceLabel}[1]{\hspace{-1.6em}\makebox[1.6em][l]{\textbf{#1}}\ignorespaces}
\newcommand{\WrongChoice}[1]{\choice \ChoiceLabel{#1}}
\newcommand{\RightChoice}[1]{\correctchoice \ChoiceLabel{#1}}
\newcommand{\Item}[1]{\hspace{-17pt}\makebox[17pt][l]{\textbf{#1.}}\ignorespaces}
\newcommand{\midmatch}{\hspace{0.75in}\underline{\hspace{0.25in}     }}


\setlength{\parskip}{0.6mm}
% \pgfplotsset{compat=1.18}

% ========================== colors =======================

\definecolor{mygray}{gray}{0.9}
\definecolor{maroon}{cmyk}{0,0.87,0.68,0.32}
\definecolor{almond}{rgb}{0.94, 0.87, 0.8}
\definecolor{cosmiclatte}{rgb}{1.0, 0.97, 0.91}
\definecolor{eggshell}{rgb}{0.94, 0.92, 0.84}
\definecolor{floralwhite}{rgb}{1.0, 0.98, 0.94}
\definecolor{darksienna}{rgb}{0.24, 0.08, 0.08}
\definecolor{darkorange}{rgb}{1.0, 0.55, 0.0}
\definecolor{cinnabar}{rgb}{0.89, 0.26, 0.2}
\definecolor{caribbeangreen}{rgb}{0.0, 0.8, 0.6}
\definecolor{aliceblue}{rgb}{0.94, 0.97, 1.0}
\definecolor{burlywood}{rgb}{0.87, 0.72, 0.53}
\definecolor{neonorange}{rgb}{255, 151, 24} 

% ========================== watermark =======================

% \usepackage[firstpage]{dr
% \SetWatermarkLightness{ 0.9 }
% \SetWatermarkColor{cinnabar!30}
% \SetWatermarkText{BOZZA}
% \SetWatermarkScale{ 1.4 }

% ---------------------------------- Intestazione

\newcommand{\class}{\LARGE {Verifica di Matematica - v.1.3}}
\newcommand{\term}{\Large {\em Trigonometria}}
% \newcommand{\recupero}{\large {\em  Valida per il recupero del debito del primo quadrimestre}}
\newcommand{\examnum}{anno scolastico 2024/'25}
\newcommand{\examtitle}{{prof. Diego Fantinelli}}
\newcommand{\examdate}{22 novembre 2024}
\newcommand{\timelimit}{100 minuti}
\CorrectChoiceEmphasis{\color{neonorange}}
\SolutionEmphasis{\color{maroon} \footnotesize}
\renewcommand{\solutiontitle}{\noindent\textbf{Soluzione:}\par\noindent}

\pagestyle{headandfoot}
\firstpageheader{}{}{}
\runningheader{\small Liceo Scientifico "J. Da Ponte" - Bassano del Grappa (VI)}{}{Classe 4 AS}
\runningheadrule

\firstpagefooter{}{}{}
\runningfooter{\footnotesize Verifica di matematica}{\footnotesize Trigonometria e appliazioni}{\thepage\ di \numpages}
\runningfootrule

% ---------------------------------- Definition of circles

\def\firstcircle{ (0.0, 0.0) circle (1.5)}
\def\secondcircle{(2.0, 0.0) circle (1.5)}
\def\thirdcircle{ (1.0,-1.5) circle (1.5)}
\def\rectangle{ (-1.5,-3.0) rectangle (3.5,1.0) }
\colorlet{circle edge}{black}
\colorlet{circle area}{blue!30}

\tikzset{filled/.style={fill=circle area, draw=circle edge, thick},
outline/.style={draw=circle edge, thick}}

\setlength{\parskip}{1.2mm}

% =============================================== Punteggi

\pointpoints{punto}{\em punti}
\pointformat{[{\footnotesize \thepoints}]}
\bonuspointpoints{punto bonus}{\em p.ti bonus}
\bonuspointformat{[{\footnotesize \thepoints}]}
\pointsinrightmargin
\setlength{\rightpointsmargin}{1.6cm}
\chqword{Esercizio}
\chpword{Punti}
\chbpword{Punti Bonus}
\chsword{Punteggio}
\chtword{Totale}

\begin{document}

% =============================================== Title Page

\tikz [remember picture, overlay] %
\node [shift={(1.2cm,-0.5cm)}] at (current page.north west) %
[anchor=north west] %
{\includegraphics[width = 26mm]{/Users/diegofantinelli/Documents/Logos/logo2.png}};
\begin{tikzpicture}[remember picture, overlay]
  \node[align=left, text width=14cm, anchor=north west] (texy)
    at ($(current page.north) + (-7cm,-0.6cm)$)
    {
      \begin{center}
        \footnotesize {\bf LICEO SCIENTIFICO STATALE "Jacopo Da Ponte"}\\  
        Via S. Tommaso D'Aquino, 12 - 36061 Bassano del Grappa (VI)\\ tel: 0424.522280 - email: \href{mailto:VIIS01700L@istruzione.it}{VIIS01700L@istruzione.it}
      \end{center}
    };
\end{tikzpicture}
% \rule[2ex]{\textwidth}{0.5pt}\\
\tikz [remember picture, overlay] %
\node [shift={(17.4cm,-0.8cm)}] at (current page.north west) %
[anchor=north west] %
{\includegraphics[width = 18mm]{/Users/diegofantinelli/Documents/Logos/logo_italy.png}};

% ================================== Title Page ================================ 
\vfill
\begin{center}
  \vfill
  \rule[2ex]{\textwidth}{0.5pt}\\
  {\huge{\bf \class}}\\[10pt]
  {\Large{ \term }}\\[8pt]
  % {\Large{ \recupero }}\\[8pt]
  % {\em \Large{Argomento: \examtitle}}\\[8pt]
  \rule[2ex]{\textwidth}{0.5pt}\\
\end{center}
\vspace{20pt}
\begin{tabular*}{\textwidth}{l @{\extracolsep{\fill}} r @{\extracolsep{6pt}} l}
  \textbf{} & \textbf{COGNOME e Nome:} & \makebox[2.5in]{\hrulefill}\\
  \textbf{} &&\\
  \textbf{} & \textbf{Classe:} & \makebox[2.5in]{\Large{\bf 4 AS}}\\
  \textbf{} &&\\
  \textbf{} & Tempo a disposizione: & \makebox[2.5in]{\timelimit}\\
  \textbf{} &&\\
  \textbf{} & data: & \makebox[2.5in]{\examdate}\\
  \textbf{} &&\\
  \textbf{} &&\\
  \textbf{} & {\em prof.:} & \makebox[2.5in]{\em Diego Fantinelli}\\
  \textbf{} &&\\
  \textbf{} &&\\
  \textbf{} &&\\
  \textbf{} & {\em voto finale:} & \makebox[2.5in]{\fillin}\\[20pt]
\end{tabular*}\\
\noindent
\faCommentO {\em \hspace{2pt} eventuali osservazioni e/o considerazioni del docente}:
\vspace{0pt}
\fillwithlines{0.50in}
\vfill

% ========================== Avvertenze =================================

\small{\textbf{Istruzioni e avvertenze:}}
\small{
  \SetSinglespace{1ex}{
    \begin{itemize}
      \item La presente verifica, somministrata in modalità {\em in presenza}, contiene \numquestions \, quesiti, per un totale di \numpoints \, punti, più un quesito bonus del valore di $2$ punti bonus nel caso in cui siano stati risolti tutti i precedenti, altrimenti varrà $1$ punto. 
        % \item Per la parte riguardante il TEST:
        % \begin{itemize}
        %     \item Le risposte vanno accuratamente riportate nella {\bf Tabella delle Risposte} allegata;
        %     \item Ogni risposta esatta vale $2$ punti;
        %     \item Le risposte che richiedono una giustificazione valgono $2$ punto soltanto se corrette e complete;
        %     \item Ogni risposta errata o non data vale $0$ punti;
        %     \item Per modificare una risposta è sufficiente cerchiare quella errata e segnare nuovamente quella corretta.
        % \end{itemize}
      \item La sufficienza è fissata a $20$ punti
        % \begin{itemize}
        %   \item {\bf nota:} per il {\bf recupero del debito del primo quadrimestre} fare riferimento alla relativa griglia di valutazione: il debito sarà considerato recuperato con un punteggio pari a $12$ punti.
        % \end{itemize}
        % \item {\bf nota:} per il {\bf recupero del debito del primo quadrimestre} fare riferimento alla relativa griglia di valutazione
      \item Il voto verrà riportato in capo alla presente verifica, e sarà oggetto di un confronto costruttivo con lo studente.
      \item Eventuali copiature palesi comporteranno l'annullamento della prova e un voto pari a 3, a prescindere dal punteggio totalizzato.
        % \item La sufficienza è fissata a $30$ punti, ma potrebbe subire delle modifiche in fase di correzione al fine di garantire la validità della prova anche in caso di prestazioni lontane dalla media-classe auspicata.
      \item È vietato l'utilizzo di calcolatrici scientifiche, smartphone, smartwatch, tablet e altri dispositivi digitali, così come l'accesso a internet, nonché la consultazione di testi, appunti e/o siti web, ove non preventivamente autorizzato.
    \end{itemize}
  }
}

\newpage
% \printanswers

% ============================== Esercizi ==============================

\subsection*{Trigonometria}
\subsection*{Teoremi sui Triangoli rettangoli e loro applicazioni}
\vspace{12pt}
\hrule
\vspace{10pt}
\begin{questions}

  % --------------------------------- Esercizio 1
  \addpoints
  \question[6]
  Risolvi il triangolo rettangolo $ABC$ sapendo che l'angolo $\gamma = 70^{\circ}$ e che $AH$ l'altezza relativa\\ all'ipotenusa misura $12 cm$

      % \fillwithdottedlines{0.5in}
      \begin{solution}
        L'angolo $\beta$ è ampio $90^{\circ}-70^{\circ}=20^{\circ}$.
Possiamo ora considerare il triangolo $A C H$, rettangolo in $H$. Applicando le relazioni trigonometriche abbiamo:
$$
b=\dfrac{A H}{\operatorname{sen} \gamma}=\dfrac{12 \mathrm{~cm}}{\operatorname{sen} 70^{\circ}}=12,77 \mathrm{~cm} .
$$

Calcoliamo ora la misura dell'ipotenusa del triangolo $A B C$ :
$$
a=\dfrac{b}{\cos \gamma}=\dfrac{12,77}{\cos 70^{\circ}}=37,34 \mathrm{~cm}
$$

Possiamo calcolare la misura del cateto $c$ in vari modi, per esempio:
$$
c=a \cdot \cos \beta=37,34 \mathrm{~cm} \cdot \cos 20^{\circ}=35,09 \mathrm{~cm} .
$$

Il triangolo $A B C$ è così risolto.
      \end{solution}
      \begin{center}
        \begin{tikzpicture}
          % Disegna la griglia quadrettata
      \draw[step=0.4cm, gray!30, very thin] (-10.0,-2.8) grid (7.2,2.8);

          % % Disegna la circonferenza
          % \draw[thick] (0,0) circle (3cm);

        \end{tikzpicture}
      \end{center}

  \vspace{12pt}
  % \hrule width \hsize \kern 0.5mm \hrule width \hsize height 1pt
  \vspace{15pt}
  \hrule

  % --------------------------------- Esercizio 2 :TODO
  \addpoints
  \question[8]
  Determina il valore della lunghezza della corda $\overline{\text{AB}}$ sapendo che l'angolo al centro vale $\alpha = 120^{\circ}$\\ e il raggio della circonferenza $R = 4 cm$

    % \fillwithdottedlines{0.5in}
    \begin{solution}
      $\left[AB = 4 \sqrt{3} \right]$
    \end{solution}
    \vspace{15pt}
    \begin{center}
    \begin{tikzpicture}
      % Disegna la griglia quadrettata
      \draw[step=0.4cm, gray!30, very thin] (-10.0,-2.8) grid (7.2,2.8);

      % Disegna la circonferenza
      % \draw[thick] (0,0) circle (3cm);

    \end{tikzpicture}
  \end{center}

\hrule
\newpage
  \subsubsection*{Risoluzione di Triangoli qualsiasi}
  \vspace{10pt}

  % --------------------------------- Esercizio 3
  % \hrule width \hsize \kern 0.5mm \hrule width \hsize height 1pt
  \hrule
  \addpoints
  \question[10]
Risolvi il triangolo $A B C$, sapendo che: $b=4 \sqrt{2}, \quad c=4 \sqrt{6}, \quad \beta=30^{\circ}$.

% \begin{center}
%     \includegraphics[width=0.9\textwidth]{trig01.png} % Cambia "nome_immagine.png" con il nome del file
% \end{center}
\vspace{10pt}
    \begin{tikzpicture}
      % Disegna la griglia quadrettata
      \draw[step=0.4cm, gray!30, very thin] (-10.0,-2.8) grid (7.2,2.8);

      % Disegna la circonferenza
      % \draw[thick] (0,0) circle (3cm);

    \end{tikzpicture}
% \fillwithdottedlines{0.5in}
      \begin{solution}
        Applichiamo più volte il teorema del coseno:
$$
\begin{aligned}
& \cos \alpha=\frac{b^2+c^2-a^2}{2 b c}=\frac{17^2+20^2-12^2}{2 \cdot 17 \cdot 20} \simeq 0,80 \rightarrow \alpha \simeq 36,72^{\circ} ; \\
& \cos \beta=\frac{a^2+c^2-b^2}{2 a c}=\frac{12^2+20^2-17^2}{2 \cdot 12 \cdot 20} \simeq 0,53 \rightarrow \beta \simeq 57,91^{\circ} ; \\
& \cos \gamma=\frac{a^2+b^2-c^2}{2 a b}=\frac{12^2+17^2-20^2}{2 \cdot 12 \cdot 17} \simeq 0,08 \rightarrow \gamma \simeq 85,36^{\circ} .
\end{aligned}
$$
      \end{solution}
  % \newpage
  \hrule

  \vspace{10pt}
  % ----------------------------------------Esercizio 5

\addpoints
  \question[4] Enuncia il Teorema del Coseno - o di Carnot - e fornisci la dimostrazione.

\fillwithdottedlines{0.5in}
    \begin{center}
    \begin{tikzpicture}
      % Disegna la griglia quadrettata
      \draw[step=0.4cm, gray!30, very thin] (-10.0,-2.8) grid (7.2,2.8);

    \end{tikzpicture}
  \end{center}
  \vspace{10pt}

  \hrule
  % \hrule width \hsize \kern 0.5mm \hrule width \hsize height 1pt

  % ----------------------------------------Esercizio 5

\addpoints
\question[4] In un triangolo l'area misura $\dfrac{\sqrt{3}}{2}(1+ \sqrt{3})$ e due angoli hanno ampiezze $\dfrac{\pi}{4}$ e $\dfrac{\pi}{3}$.\\
Determina le misure degli altri elementi del triangolo.
% \fillwithdottedlines{0.75in}
    \begin{center}
    \begin{tikzpicture}
      % Disegna la griglia quadrettata
      \draw[step=0.4cm, gray!30, very thin] (-10.0,-2.8) grid (7.2,2.8);

    \end{tikzpicture}
  \end{center}
  \vspace{10pt}
  % \hrule width \hsize \kern 0.5mm \hrule width \hsize height 1pt
  \newpage

  % ----------------------------------------Esercizio 6

  \hrule
\addpoints
\question[4] In un triangolo l'area misura $\dfrac{\sqrt{3}}{2}(1+ \sqrt{3})$ e due angoli hanno ampiezze $\dfrac{\pi}{4}$ e $\dfrac{\pi}{3}$.\\
% \fillwithdottedlines{0.75in}
    \begin{center}
    \begin{tikzpicture}
      % Disegna la griglia quadrettata
      \draw[step=0.4cm, gray!30, very thin] (-10.0,-2.8) grid (7.2,2.8);

    \end{tikzpicture}
  \end{center}
  \vspace{10pt}
  % ========================================= Esercizio 6 - Facoltativo
  \vfill
  \rule[.1ex]{\textwidth}{0.3pt}
  \bonusquestion[2] {\em Esercizio facoltativo:} Enuncia il Teorema del Coseno - o di Carnot - e fornisci la dimostrazione.

  \vspace{10pt}
    \begin{center}
    \begin{tikzpicture}
      % Disegna la griglia quadrettata
      \draw[step=0.4cm, gray!30, very thin] (-10.0,-2.8) grid (7.2,2.8);

    \end{tikzpicture}
  \end{center}
  \vspace{10pt}
  \vspace{15pt}
  % \hrule width \hsize \kern 0.5mm \hrule width \hsize height 1pt
  % \hrule
\end{questions}
\newpage

\vfill
% ================================ Tabella Punteggi
% \vfill
\begin{center}
  \subsubsection*{tabella dei punteggi}
  \combinedgradetable[v][questions]

  \footnotesize{la sufficienza è fissata a $20$ punti}

  \subsubsection*{Conoscenze, abilità e competenze}

  % Cambia font solo per la parte che segue
  {\fontfamily{ptm}\selectfont  % ptm è il codice per Times New Roman
    \footnotesize
    \begin{tabular}{|c|c|c|c|}
      \hline 
    & \textbf{conoscenze} & \textbf{abilità} & \textbf{competenze} \\
    \hline 
      \textbf{eccellente} & 5 & 3 & 2 \\
      \hline 
      \textbf{ottimo} & 4.5 & 2.75 & 1.75 \\
      \hline 
      \textbf{buono} & 4 & 2.5 & 1.5 \\
      \hline 
      \textbf{discreto} & 3.5 & 2.25 & 1.25 \\
      \hline 
      \textbf{sufficiente} & 3 & 2 & 1 \\
      \hline 
      \textbf{quasi sufficiente} & 2.75 & 1.875 & 0.875 \\
      \hline 
      \textbf{insufficiente} & 2.5 & 1.75 & 0.75 \\
      \hline 
      \textbf{gravemente insufficiente} & 2 & 1.5 & 0.5 \\
      \hline 
      \textbf{scarso} & 1.5 & 1.25 & 0.25 \\
      \hline
    \end{tabular}
    \vspace{2pt}

    \footnotesize{$^{*}$Per gli indicatori e i descrittori si fa riferimento a quelli esplicitati nella programmazione.\\ 
    ciascun valore espresso nella tabella va inteso come massimo dei punti attribuibili.}
  }
\end{center}
\noindent

% \hrule width \hsize \kern 0.5mm \hrule width \hsize height 1pt
\begin{center}
  \subsubsection*{Griglia di valutazione}
\end{center}

% \begin{multicols}{2}

{\color{cinnabar}
  \begin{center}
    % \bf{Verifica}
  \end{center}
  \begin{center}
    \begin{tabular}{c|c}
      \toprule

      \bf{punteggio} &  \bf{voto}\\[5pt]
      \midrule
      $ < 8$ & $3\frac{1}{2}$ \\[3pt]

      $8$ & $4$ \\[3pt]

      $10$ & $4\frac{1}{2}$ \\[3pt]

      $14$ & $5$ \\[3pt]

      $18$ & $5 \frac{1}{2}$ \\[3pt]

      \large {$20$} & \large{$6$} \\[3pt]

      $24$ & $6 \frac{1}{2}$ \\[3pt]

      $26$ & $7$ \\[3pt]

      $30$ & $7 \frac{1}{2}$ \\[3pt]

      $32$ & $8$ \\[3pt]

      $36$ & $8 \frac{1}{2}$ \\[3pt]

      $38$ & $9$ \\[3pt]

      $40$ & $9 \frac{1}{2}$ \\[3pt]

      $40 - \text{\em bonus}$ & $10$ \\[5pt]

      \bottomrule
    \end{tabular}
  \end{center}
}

% \newpage


% \begin{tikzpicture}
% Disegna la circonferenza
% \draw[thick] (0,0) circle (2cm);

% Disegna gli assi
% \draw[->, thick] (-3.5,0) -- (3.5,0) node[anchor=north west] {$x$};
% \draw[->, thick] (0,-3.5) -- (0,3.5) node[anchor=south east] {$y$};

% Aggiungi i punti fondamentali
% \filldraw[black] (3,0) circle (2pt) node[anchor=north] {1};
% \filldraw[black] (-3,0) circle (2pt) node[anchor=north] {-1};
% \filldraw[black] (0,3) circle (2pt) node[anchor=east] {1};
% \filldraw[black] (0,-3) circle (2pt) node[anchor=west] {-1};

% Aggiungi angoli principali
% \draw[thick, ->] (0,0) -- (2.12,2.12) node[midway, anchor=south west] {$\frac{\pi}{4}$};
% \draw[thick, ->] (0,0) -- (0,3) node[midway, anchor=west] {$\frac{\pi}{2}$};
% \draw[thick, ->] (0,0) -- (-2.12,2.12) node[midway, anchor=south east] {$\frac{3\pi}{4}$};
% \draw[thick, ->] (0,0) -- (-3,0) node[midway, anchor=north] {$\pi$};
% \draw[thick, ->] (0,0) -- (-2.12,-2.12) node[midway, anchor=north east] {$\frac{5\pi}{4}$};
% \draw[thick, ->] (0,0) -- (0,-3) node[midway, anchor=east] {$\frac{3\pi}{2}$};
% \draw[thick, ->] (0,0) -- (2.12,-2.12) node[midway, anchor=north west] {$\frac{7\pi}{4}$};

% Etichetta origine
% \node at (0,0) [anchor=north east] {O};
% \end{tikzpicture}
\end{document}


